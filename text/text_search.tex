\chapter{Textual Search}

The model described in Chapter \ref{} assigns each image $\bm{x}$ a vector $\bm{\hat{y}}$. In theory, it represents how likely each class is contained in the image. However, this representation is far from useful for a user. Therefore multiple techniques are used to ensure the model is convenient yet powerful search tool even for novices.

\section{Supported Labels}
Our tool supports only a finite set of labels $L$ that are prompted by the interface when users form the query (see Chapter \ref{}). It could be possible to construct a query given higher level description transformed by task-specific set of neural language processing rules \cite{moumtzidou2017verge}, but this approach may mislead user if a searched high level concept is not in an internal model. Thus a set $L_m$ is created containing labels corresponding to the image classes with their \textsf{names}, \textsf{descriptions} and \textsf{hyponyms} taken from WordNet \cite{WordNet}. We further utilize WordNet structure creating a new set $L_h$ of hypernyms of all labels in $L_m$. The final set of labels is then $L=L_m\cup L_h$. With hypernym--hyponym relation as a directed edge, $L$ can be viewed as a directed acyclic graph (DAG) where all vertices (i.e. labels) with no outgoing edges are in $L_m$ (however, there can be vertices in $L_m$ with outgoing edges).

\section{Query Formulation and Ranking}\label{sec:query_formulation_and_ranking}
The users are allowed to specify sets of supported labels $N_i \subseteq L$, where in each set the labels are connected by logical \textsf{OR}, while the sets $N_i$ are connected by logical \textsf{AND}. For $k$ such sets, the user query $Q_u$ is then written as
\begin{equation}
	Q_u=\bigwedge\limits_{i=1}^k\left(\bigvee\limits_{\forall label_j\in N_i} label_j \right)
\end{equation}
and for convenience it is usually abbreviated as $Q_u=\{N_i\}_{i=1}^k$. If the query contains any hypernyms (labels with some outgoing edges in $L$), further preprocessing needs to be done. By default every hypernym $h$ is substituted by a set of all labels in $L_m$ that are reachable from $h$ in DAG $L$. For hypernyms in $L_m$ (i.e. they correspond to a class recognized by the model), the substitution can be disabled. Finally, the preprocessed query $Q_p$ contains only labels from $L_m$ directly recognized by our model.

The ranking $r(\cdot)$ for each image $\bm{x}$, given preprocessed query $Q_p$ and model parametrized by $\bm{\theta}$, is calculated as 
\begin{equation}
r\left(\bm{x}; Q_p, \bm{\theta}\right)=\prod\limits_{\forall N_i \in Q_p}\left(
	\sum\limits_{\forall label_j\in N_i} \bm{\hat{y}}_{label_j}\cdot idf\left(label_j\right)
\right)\label{eq:text_rank}
\end{equation}
where $\bm{\hat{y}}=f\left(\bm{x}, \bm{\theta}\right)$  is model prediction on the image $\bm{x}$ and $\bm{\hat{y}}_{label_j}$ represents its relevance score of containing $label_j$ and $idf(\cdot)$ represents inverse document frequency (IDF). Let us explain underlying reasoning why to introduce IDF on an example: When a user searches \textit{a person in front of an excavator}, person keyword completely dominates the query since in the dataset there are thousand times as many persons as excavators. Our decision is also supported by simulations in Chapter \ref{} where queries with IDF rank searched images higher. Standard IDF however uses number of documents over number of documents containing given term which is unusable because softmax layer gives small nonzero values to almost all classes. We thus define inverse document frequency as
\begin{equation}
idf(label) = \log\left(
\frac{
\max\limits_{i\in labels} \sum_{\bm{x}}\hat{\bm{y}}_{i}
}{
\sum_{\bm{x}}\hat{\bm{y}}_{label}
} + 1\right)\label{eq:idf}
\end{equation}
where $\hat{\bm{y}}$ is the same as in equation \ref{eq:text_rank}, therefore dependent on image $\bm{x}$. If we thing of $\hat{\bm{y}}_i$ as an (unnormalized) probability image $\bm{x}$ contains label $i$, the fraction in equation \ref{eq:idf} can be viewed as probability that any given image contains the most common label in a collection over probability it contains the searched label.

\section{Application Prototype}

\section{Other Retrieval Models}
Our work is part of bigger project focused on video retrieval with goal of creating a tool that provides user with powerful means to interactively search his video collection. In this section we therefore summarize our tool's other retrieval models and model fusion used for Lifelog Search Challenge (LSC) workshop at ICMR2018~\cite{LokocLSC} and for Video Browser Showdown (VBS) competition at MMM2018~\cite{lokovc2018revisiting}.

Mainly for visual KIS tasks our tool features easy-to-use canvas for color sketches. Every sketch is represented as a set $Q_c = \{\left(q_i, r_i, t_i\right)\}^k_{i=1}$ of $k$ color points where $q_i\in\R^3$ is color in CIE Lab color space, $r_i$ represents region where the given color shall be located and $t_i\in \{\mathrm{ALL},\mathrm{ANY}\}$ specifies whether ALL or just ANY of the pixels in the region should be considered. The ranking $c(\cdot)$ of an image $\bm{x}$ is then computed as:
\begin{equation}
c\left(\bm{x}; Q_c\right) =-\left(\sum\limits_{\substack{\forall \left(q_i,r_i,t_i\right) \in Q_c\\t_i=\mathrm{ANY}}} \min\limits_{p\in\bm{x}\,\mathrm{in}\,r_i}L_2\left(q_i, p\right)+\sum\limits_{\substack{\forall \left(q_i,r_i,t_i\right) \in Q_c\\t_i=\mathrm{ALL}}} \mathop{\mathrm{avg}}\limits_{p\in\bm{x}\,\mathrm{in}\,r_i}L_2\left(q_i, p\right)\right)
\end{equation}
where $p\in\bm{x}\ \mathrm{in}\ r_i$ represents all pixels of image $\bm{x}$ in specified region $r_i$. Due to performance reasons each image in database is represented only as $20\times 15$ color points or `pixels'. Our user interface also currently supports regions in shape of ellipses only. If keyword or color-sketch search is not successful or user wants to simply browse similar images, our tool utilizes deep features from GoogLeNet~\cite{szegedy2015going} after the last average pooling layer for retrieval of semantically similar images. Rank of an image given a query example is the cosine similarity between their deep feature vectors.


All the retrieval models provide a relevance score function inducing a similarity based ordering of all database objects with respect to a given query. Even though our tool supports multiple query modalities at once by normalizing ranking of each modality to interval $[0, 1]$ and then summing them up, combining multiple models can be tricky. This is because each model has different distribution of rank values, e.g. similarity ranking assigns to most of the database rank between $0.15$ and $0.4$ whereas in keyword ranking almost all images are assigned rank smaller than~$10^{-2}$. However our tool enables us to use other models as filters where threshold can be dynamically set for each model interdependently. Such approach proved to be much more useful than sorting joined ranking from multiple models.

